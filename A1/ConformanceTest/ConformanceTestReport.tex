\documentclass[ieee]{submit}

\title{State-Based Testing}
\date{2015-10-05}
\author{
	\MakeUppercase{Neil Edelman}
	\affil{McGill University (110121860)}
	\MakeUppercase{Casimir D\'esarmeaux}
	\affil{McGill University}
}
\setinfo[A1]{N. Edelman, C. D\'esarmeaux}{\thetitle}{304}{429}{001}{Fall}{2015}

\begin{document}

\category{D.1.3}{Software}{Concurrent Programming}
\terms{Software}
\keywords{Subject}

\begin{abstract}
Assignment 1 report, part 1.
\end{abstract}

\maketitle

\course{Prof. G. Mussbacher}{ECSE 429 -- Introduction to Software Quality Assurance}

\clearpage
\pagenumbering{roman}

\tableofcontents

%\hr

%\listoffigures

\clearpage
\pagenumbering{arabic}

\section{Source Code}

\subsection{Source code}

\begin{quote}
The source code that generates the test class for the CCoinBox example given the state
machine definition and implementation of the state machine.
\end{quote}

The code is in the {\tt package ca.mcgill.ecse429.conformancetest.nplus}.

\subsection{Generated code}

\begin{quote}
The result of the test class generation for the CCoinBox example without any manual changes
after generation. This class must be called GeneratedTestCCoinBox.java and saved in the same package as the implementation of the state machine.
\end{quote}

The generated code is in {\tt test}, following the same directory structure.

\section{Complete Test Class}

\begin{quote}
The complete test class for the CCoinBox example with additional code added manually as
needed to fully test the CCoinBox state machine based on the N+ Test Strategy (conformance tests only). This class must be called TestCCoinBox.java and saved in the same package as the implementation of the state machine. Any manual changes have to be clearly identified in the
complete test class. Any complete test class that cannot be executed as a JUnit test will result in a mark of 0 for this part.
\end{quote}

The generated code is in {\tt test}, following the same directory structure.

\section{Report}



\begin{itemize}
\item List.
\item List.
\end{itemize}

\begin{comment}
\wide[height=0.98\textheight]{rtl-flash}{{\tt g49\_flash\_read\_control}: flash circuit.}

\begin{figure*}[!ht]\centering
	\subfloat[state machine]{%
		\includegraphics[width=0.96\textwidth]{state2.jpg}
		\label{control:1}
	}
	\\
	\subfloat[transitions]{%
		\includegraphics[width=0.7\textwidth]{state2-tran.jpg}
		\label{control:2}
	}
	\caption{The state machine diagram for {\tt g49\_mid.vhd}.}
	\label{machine}
\end{figure*}

\begin{table}[htb]
\begin{center}\begin{tabular}{lr@{.}l}%{|lr@{.}l|}
%\hline
$F_{\text{max}}$& 50&69\,MHz \\%50.71\,MHz\\
Setup& 0&272\,ns\\%0.280\\
Hold& 0&445\,ns\\
%\hline
\end{tabular}\end{center}
\caption{{\tt g49\_mid.vhd}: timing values for the EP2C20F484C8N.}
\label{midtiming}
\end{table}

\begin{algorithm*}[h]
\DontPrintSemicolon
\KwIn{$A$}
\KwOut{$B$}
\BlankLine
\ForEach{$a \in A$}{
	\If{$a$}{$B$}
}
\KwRet{$B$}\;
\label{test}
\caption{test}
\end{algorithm*}

\begin{align} % simmons tunneling
I(V)&=\frac{q^{2}}{4\pi^{2} \hbar d^{2}}\left(
\left( \phi -\alpha qV \right)
e^{-K\sqrt{\phi -\alpha qV}}\right.\nonumber\\
&\quad\left.-\left( \phi +\left( 1-\alpha \right)qV \right)
e^{-K\sqrt{\phi +\left( 1-\alpha  \right)qV}}
\right)\nonumber\\
&\text{where, }\nonumber\\
K&=\frac{2d}{\hbar}\sqrt{2m_{e}q}\nonumber\\
&\text{provided, }\nonumber\\
V &< \frac{\phi}{q}\label{sim}
\end{align}

\bibliography{g49_lab_5}

\end{comment}

\received{\thedate}{\thedate}{\thedate}

%\clearpage
%\pagenumbering{withappendix}

%\appendix
%\section*{APPENDIX-1}

%Appendix.

\end{document}
